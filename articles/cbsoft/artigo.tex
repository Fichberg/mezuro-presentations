\documentclass[12pt]{article}

\usepackage{sbc-template}
\usepackage{graphicx,url}
\usepackage[brazil]{babel}
\usepackage[utf8]{inputenc}

\usepackage{tabularx}

\sloppy

\title{Como a Adoção de Projetos de Software Livre Como Base para Novos Projetos Afeta Seu Desenvolvimento}

\author{Rafael R. Manzo\inst{1}, Diego de A. M. Camarinha\inst{1},\\
        Guilherme H. R. V. de Lima\inst{1}, Fellipe S. Sampaio\inst{1},\\
        Renan Fichberg\inst{1}, Paulo Meirelles\inst{2}}


\address{Instituto de Matemática e Estatística -- Universidade de São Paulo (USP)\\
  Rua do Matão, 1010 -- 05508-090 -- Cidade Universitária -- São Paulo -- SP -- Brasil
\nextinstitute
  Faculdade de Engenharia -- UnB Gama (FGA)\\
  Gama -- DF -- Brasil
  \email{manzo@ime.usp.br,\{diego.camarinha,guilherme.henrique.lima\}@usp.br}
  \email{\{renan.fichberg,fellipe.sampaio\}@usp.br,paulo@softwarelivre.org}
}

\begin{document}

\maketitle

\begin{abstract}
  This resumed article approaches one of the most important decisions when starting a new free software project: to adopt and base the new project on an old and mature one or to create a whole new and independent one. The vantages and disadvantages for each option will be discussed for, at last, present a real experience from a project that started by choosing the second option, during it's early development switched to the first one, to recently got back again to the second one.
\end{abstract}

\begin{resumo}
  Este artigo resumido trata sobre uma das decisões mais importantes no momento de se iniciar um novo projeto de software livre: adotar como base outro projeto já maduro ou escrever um totalmente novo e independente. Serão discutidas brevemente as vantagens e desvantagens de cada opção para, por fim, apresentar uma experiência real de um projeto que teve início decidindo pela segunda opção, durante o início de seu desenvolvimento resolveu adotar a primeira opção, para recentemente novamente adotar a segunda opção.
\end{resumo}


\newpage
\bibliographystyle{sbc}
\bibliography{impacto-software-livre}

\end{document}
