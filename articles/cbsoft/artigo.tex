\documentclass[12pt]{article}

\usepackage{sbc-template}
\usepackage{graphicx,url}
\usepackage[brazil]{babel}
\usepackage[utf8]{inputenc}

\usepackage{tabularx}

\sloppy

\title{Como a Adoção de Projetos de Software Livre Como Base para Novos Projetos Afeta Seu Desenvolvimento}

\author{Rafael R. Manzo\inst{1}, Diego de A. M. Camarinha\inst{1},\\
        Alessandro Palmeira\inst{1}, Fellipe S. Sampaio\inst{1},\\
        Renan Fichberg\inst{1}, Paulo Meirelles\inst{2}}


\address{Instituto de Matemática e Estatística -- Universidade de São Paulo (USP)\\
  Rua do Matão, 1010 -- 05508-090 -- Cidade Universitária -- São Paulo -- SP -- Brasil
\nextinstitute
  Faculdade de Engenharia -- UnB Gama (FGA)\\
  Gama -- DF -- Brasil
  \email{manzo@ime.usp.br,\{diego.camarinha,alessandro.palmeira\}@usp.br}
  \email{\{renan.fichberg,fellipe.sampaio\}@usp.br,paulo@softwarelivre.org}
}

\begin{document}

\maketitle

\begin{resumo}
  ESCREVER NO FINAL BASEADO NO QUE FALARMOS AO LONGO DESTE ARQUIVO!
\end{resumo}

\section{Mezuro: o que é?} \label{sec:projeto-mezuro}
O projeto Mezuro (\url{http://mezuro.org}) com forte viés acadêmico, visa ser uma interface que permita, de forma flexível, a extração e análise de métricas estáticas de código fonte. O usuário é o responsável por definir o conjunto de métricas a ser utilizado para realizar cálculos, com a possibilidade de armazenar os resultados para comparações futuras. Seu objetivo é: 
\begin{itemize}
    \item Aproximar-se de um consenso acerca de quais métricas devem ser empregadas na análise da qualidade de um código fonte
    \item Buscar os valores destas métricas que definem a qualidade deste código.
\end{itemize}

\subsection{Por que surgiu?} \label{subsec:problemas}
As principais motivações para o surgimento de uma ferramenta como o Mezuro são os seguintes problemas:
\begin{itemize}
    \item Não há parâmetros de comparação consolidados
    \item Existem estudos, mas poucos dados empíricos
    \item Ainda é dada pouca importância ao monitoramento de código
\end{itemize}

\section{Por que usar o Mezuro?} \label{sec:projeto-mezuro}
Idealizado como um plataforma de métricas de código, um dos diferenciais do Mezuro reside na possibilidade de gerar informação sobre o código fonte de forma contínua, o usuário decide quando analisar novamente o projeto e acompanha detalhadamente a evolução  das notas ao longo do tempo. Os resultados de cada análise é pública, o que permite uma maior transparência entre o desenvolvedor e a comunidade que utiliza aquele software, esta, que através dos resultados das métricas providas pelo Mezuro, pode decidir se aquela solução atende ou não as suas necessidades e se deve depositar confiança na qualidade do software desenvolvido.

\section{Principais funcionalidades}\label{sec:princ-funcionalidades}
No Mezuro as funcionalidades podem ser divididas em dois grupos, projeto e configuração, que são:
\begin{itemize}
  \item Projeto
    \begin{itemize}
    \item Download do código-fonte a partir de repositórios (Git, Subversion, Bazaar, etc...) ou via arquivo compactado.
        \item Escolha da peridicidade do processamento do código (1 dia, 2 dias, semanal, quinzenal e mensal).
        \item Escolha de qual configuração cada repositório irá utilizar.
        \item Nota de cada métrica da configuração para cada arquivo do repositório.
        \item Análise gráfica de cada arquivo do repositório através de um gráfico de pontos com notas ao longo do tempo.
        \item Resultados públicos e acessíveis a comunidade.
    \end{itemize}
    \item Configuração
    \begin{itemize}
    \item Criação de configuração e a possibilidade de copiar de outros usuários.
        \item Estatísiticas sobre as configurações mais populares dentro da comunidade.
        \item Criação de intervalos qualitativos associados aos valores das métricas.
        \item Criação de grupos de leitura para a interpretação textual dos resultados das métricas.
        \item Combinações de métricas nativas para criação de análises compostas e mais complexas.
    \end{itemize}
\end{itemize}

\section{A rede social}\label{sec:user-potencial}
O Mezuro se apresenta a seus usuários no formato de uma rede social, no qual os participantes podem ver a produção de terceiros através da avaliação dos projetos ou do clone das configurações. O modelo de rede social foi pensado para que uma comunidade de programadores possa interagir entre si, mesmo que um dado usuário tenha experiência em métricas de software este sempre pode se basear no trabalho de outros usuários. Essa interação mútua e aberta pode ser interessante para desenvolvedores, gerentes de projeto, auditores de software e até mesmo uma equipe de desenvolvimento inteira. O objetivo final é criar uma comunidade que veja o valor de tais metodologias e como isso pode contribuir para o sucesso do seu projeto.
\newpage
\bibliographystyle{sbc}
\bibliography{impacto-software-livre}

\end{document}
