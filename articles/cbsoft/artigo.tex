\documentclass[12pt]{article}

\usepackage{sbc-template}
\usepackage{graphicx,url}
\usepackage[brazil]{babel}
\usepackage[utf8]{inputenc}

\usepackage{tabularx}

\sloppy

\title{Como a Adoção de Projetos de Software Livre Como Base para Novos Projetos Afeta Seu Desenvolvimento}

\author{Rafael R. Manzo\inst{1}, Diego de A. M. Camarinha\inst{1},\\
        Alessandro Palmeira\inst{1}, Fellipe S. Sampaio\inst{1},\\
        Renan Fichberg\inst{1}, Paulo Meirelles\inst{2}}


\address{Instituto de Matemática e Estatística -- Universidade de São Paulo (USP)\\
  Rua do Matão, 1010 -- 05508-090 -- Cidade Universitária -- São Paulo -- SP -- Brasil
\nextinstitute
  Faculdade de Engenharia -- UnB Gama (FGA)\\
  Gama -- DF -- Brasil
  \email{manzo@ime.usp.br,\{diego.camarinha,alessandro.palmeira\}@usp.br}
  \email{\{renan.fichberg,fellipe.sampaio\}@usp.br,paulo@softwarelivre.org}
}

\begin{document}

\maketitle

\begin{resumo}
  ESCREVER NO FINAL BASEADO NO QUE FALARMOS AO LONGO DESTE ARQUIVO!
\end{resumo}

\section{Mezuro: o que é?} \label{sec:projeto-mezuro}
O projeto Mezuro (\url{http://mezuro.org}) com forte viés acadêmico, visa ser uma interface que permita, de forma flexível, a extração e análise de métricas estáticas de código fonte. O usuário é o responsável por definir o conjunto de métricas a ser utilizado para realizar cálculos, com a possibilidade de armazenar os resultados para comparações futuras. Seu objetivo é: 
\begin{itemize}
    \item Aproximar-se de um consenso acerca de quais métricas devem ser empregadas na análise da qualidade de um código fonte
    \item Buscar os valores destas métricas que definem a qualidade deste código.
\end{itemize}

\subsection{Por que surgiu?} \label{subsec:problemas}
As principais motivações para o surgimento de uma ferramenta como o Mezuro são os seguintes problemas:
\begin{itemize}
    \item Não há parâmetros de comparação consolidados
    \item Existem estudos, mas poucos dados empíricos
    \item Ainda é dada pouca importância ao monitoramento de código
\end{itemize}

\section{Por que usar o Mezuro?} \label{sec:projeto-mezuro}

\newpage
\bibliographystyle{sbc}
\bibliography{impacto-software-livre}

\end{document}
