\documentclass[12pt]{article}

\usepackage{sbc-template}
\usepackage{graphicx,url}
\usepackage[brazil]{babel}
\usepackage[utf8]{inputenc}

\usepackage{tabularx}

\sloppy

\title{Como a Adoção de Projetos de Software Livre Como Base para Novos Projetos Afeta Seu Desenvolvimento}

\author{Rafael R. Manzo\inst{1}, Diego de A. M. Camarinha\inst{1},\\
        Guilherme H. R. V. de Lima\inst{1}, Fellipe S. Sampaio\inst{1},\\
        Renan Fichberg\inst{1}, Paulo Meirelles\inst{2}}


\address{Instituto de Matemática e Estatística -- Universidade de São Paulo (USP)\\
  Rua do Matão, 1010 -- 05508-090 -- Cidade Universitária -- São Paulo -- SP -- Brasil
\nextinstitute
  Faculdade de Engenharia -- UnB Gama (FGA)\\
  Gama -- DF -- Brasil
  \email{manzo@ime.usp.br,\{diego.camarinha,guilherme.henrique.lima\}@usp.br}
  \email{\{renan.fichberg,fellipe.sampaio\}@usp.br,paulo@softwarelivre.org}
}

\begin{document}

\maketitle

\begin{abstract}
  This resumed article approaches one of the most important decisions when starting a new free software project: to adopt and base the new project on a old and mature one or to create a whole new and independent one. The vantages and disadvantages for each option will be discussed for, at last, present a real experience from a project that started by choosing the second option, during it's early development switched to the first one, to recently got back again to the second one.
\end{abstract}

\begin{resumo}
  Este artigo resumido trata sobre uma das decisões mais importantes no momento de se iniciar um novo projeto de software livre: adotar como base outro projeto já maduro ou escrever um totalmente novo e independente. Serão discutidas brevemente as vantagens e desvantagens de cada opção para, por fim, apresentar uma experiência real de um projeto que teve início decidindo pela segunda opção, durante o início de seu desenvolvimento resolveu adotar a primeira opção, para recentemente novamente adotar a segunda opção.
\end{resumo}


\section{Introdução} \label{sec:introducao}
Reúso de software é uma resposta que surgiu em 1968 como solução para a crise de software \cite{nr68}. Desde então este tem sido um tópico recorrente entre toda a comunidade de computação, culminando em 1992, por Krueger, em uma compilação de toda a pesquisa feita sobre critérios de classificação para metodologias de reúso e análise de sua eficácia \cite{k92}.

Nas últimas duas décadas o reúso por meio de bibliotecas externas, ou caixa-preta \cite{hdghi11}, consolidou-se em linguagens de programação modernas como Ruby (através da plataforma \url{http://rubygems.org}) e Java (por meio do empacotamento de bibliotecas reutilizáveis em arquivos jar).

Porém, a reutilização de softwares funcionais inteiros ainda é uma questão em aberto, uma vez que exige reescrever partes do software ou que partes relevantes deste tenham as devidas abstrações que permitam a extensão de suas funcionalidades.

  \subsection{Definição de reúso de software} \label{subsec:definicaoreuso}
  Consiste de utilizar artefatos de um software já existente durante a construção de um novo sistema de software \cite{k92}.

  Estes artefatos podem ser desde trechos de código copiados que são chamados durante a execução do programa, passando por bibliotecas caixa-preta, até abstrações fornecidas por softwares como arquitetura de \textit{plugins}.

  \subsection{Critérios de classificação} \label{subsec:criteriosclassificacao}
  Conforme descrito por Krueger, há quatro critérios básicos relevantes quando para classificar o reúso de software:

  \begin{itemize}
    \item \textbf{Abstração}: a interface fornecida pelo software para que possa ser reutilizado;
    \item \textbf{Seleção}: forma de localizar, entender, comparar e escolher o que reaproveitar;
    \item \textbf{Especialização}: personalização do trecho a ser reaproveitado por meio de parâmetros;
    \item \textbf{Integração}: interface fornecida pelo software que esconde detalhes internos deste.
  \end{itemize}

\section{Decisão ao se iniciar um novo projeto} \label{sec:decisao}
Sempre que se começa um novo projeto de software surge a pergunta sobre adotar um software já existente, que abranja alguns dos requisitos e possa ser estendido com algum custo para os demais, ou escrever um novo software e arcar com os diversos riscos que isto envolve.

A seguir vamos tentar brevemente apontar vantagens e desvantagens para cada uma destas duas possibilidades afim de que possamos, após o relato de caso (\ref{sec:projeto-mezuro}), retomar estes pontos e analisar cada decisão tomada durante a vida do projeto a ser descrito.

  \newpage
  \subsection{Projeto existente como base} \label{subsec:projeto-existente}
  \begin{table}[ht]
    \centering
    \caption{Lista de vantagens e desvantagens para adoção de um software já existente em ordem decrescente de dificuldade de resolução}
    \label{tab:projeto-existente}

    \begin{tabularx}{\textwidth}{| X | X | X |}
      \hline
      \textbf{Característica} & \textbf{Vantagem} & \textbf{Desvantagem} \\ \hline
      Terceiros são responsáveis por boa parte do código &
        Custo de manutenção reduzido &
        Estar a mercê de terceiros que podem abandonar ou reescrever o projeto \\ \hline
      Tempo de desenvolvimento &
        Com boa abstração ele é reduzido drasticamente &
        Sem boa abstração, o tempo gasto para entender o software pode ultrapassar o necessário para desenvolver um novo ou até impedir por completo o desenvolvimento \\ \hline
      Credibilidade &
        Ter seu software associado a um segundo já consolidado na comunidade inspira maior confiança nos usuários &
        Por outro lado, problemas no software escolhido para ser reaproveitado, mesmo que não afetem sua extensão, trarão má publicidade para seu software\\ \hline
      Confiabilidade &
        Estar baseado em um software confiável claramente trará confiabilidade para seu software &
        Se o software base apenas aparentar ser confiável, no futuro seu software apresentará os mesmos problemas\\ \hline
    \end{tabularx}
  \end{table}

  Da tabela \ref{tab:projeto-existente} podemos ver que a diferença entre uma característica se tornar uma vantagem ou uma desvantagem consiste basicamente do tamanho do conhecimento prévio que você tem sobre o software no qual pretende se basear.

  Os três primeiros itens da tabela são especialmente complicados por dependerem de profundo conhecimento do software e das pessoas envolvidas em seu desenvolvimento.

  Isto é, a responsabilidade de terceiros por parte do código depende exclusivamente destas pessoas e suas opiniões. O melhor que pode ser feito neste caso é procurar conhecer a reputação delas. Da mesma forma, a qualidade de abstrações e a credibilidade, como segurança, de um software vão depender de conhecimento de seu código fonte mais do que tudo.

  \subsection{Projeto totalmente novo} \label{subsec:projeto-novo}
  A grande vantagem nessa opção se resume em liberdade. A qual abrange escolha de tecnologias, padrões adotados, metodologia, filosofia e todos os aspectos envolvidos no desenvolvimento de um software.

  Ou seja, é um grande poder. E, como se popularizou dizer, com grandes poderes vêm grandes responsabilidades. Então, todas estas decisões devem ser tomadas por pessoas qualificadas, experientes e conhecedoras do que tem se passado na comunidade.

  \subsection{Qual opção escolher} \label{subsec:qualopcaoescolher}
  Embora quiséssemos apresentar uma resposta exata para a pergunta, a verdade é que não existe resposta exata. Ela é subjetiva e depende de conhecimento profundo de muitas variáveis que, muitas vezes, não importa o tempo investido, não será possível conhecê-las. Um exemplo é o caso de tentar prever o comportamento de um ser humano perante um projeto de software pelo o qual este é responsável.

  Decidir por criar um software totalmente novo desde o início é uma decisão que deve ser tomada se não houverem opções de softwares parecidos existentes disponíveis ou se um especialista estiver disponível para liderar o desenvolvimento e afirmar que esta é a melhor escolha.

  Uma boa estratégia de escolha de software base é pesquisar rapidamente sobre candidatos que atendam boa parte dos requisitos do software a ser desenvolvido, procurar conhecer a eles e às pessoas envolvidas em seu desenvolvimento e, por fim, entender que tipos de abstrações estes possuem. Se existir um candidato satisfatório, apresente-o à sua equipe, ouça o que têm a dizer e com seu apoio adote-o. A cada ciclo de desenvolvimento, juntos, procurem levantar no que o software base tem sido fundamental e no que ele tem atrasado, ou até impedido, o desenvolvimento. Caso seja atingido um ponto onde não é mais possível continuar com o software base, do ponto de vista técnico, isto não será um problema, pois seu time terá acumulado a experiência necessária para desenvolver um software novo.

  Esta estratégia é falha do ponto de vista do tempo necessário. Mas ela foi recomendada porque escolhendo desde o inicio desenvolver um software novo sem conhecimento prévio suficiente da área é um risco grande comparado a adoção de um software que reúne o conhecimento de toda uma comunidade qualificada.

\section{Projeto Mezuro} \label{sec:projeto-mezuro}
Em produção em \url{http://mezuro.org}, o projeto Mezuro com forte viés acadêmico, visa ser uma interface que permita, de forma flexível, a extração e análise de métricas estáticas de código fonte \cite{wcmcww95}. O próprio usuário fica encarregado de definir qual conjunto de métricas deseja utilizar, permitindo que faça cálculos utilizando estes valores e mantendo resultados históricos para comparação futura.

Com isso, seu objetivo é o de se aproximar de um consenso sobre quais métricas devem ser empregadas na análise da qualidade de um código fonte e quais são os valores destas que definem sua qualidade.

  \subsection{História} \label{subsec:historia}
  O projeto surgiu em 2010 dentro da disciplina de Laboratório de Programação Extrema oferecida no Instituto de Matemática e Estatística da Universidade de São Paulo (IME - USP), por um time de cinco alunos. Estes decidiram por criar um software novo, englobando desde autenticação de usuários até a extração e análise de métricas de código fonte, fazendo uso do arcabouço Ruby on Rails, que na época começava a ganhar muita força dentro da comunidade com sua segunda versão. Seu código fonte pode ser encontrado em \url{https://github.com/mezuro/mezuro-sketch}.

  Esta versão falhou no quesito de viabilidade. O arcabouço adotado ainda não era maduro como a versão atual, então o time passou boa parte do tempo no desenvolvimento de funcionalidades que eram fundamentais, mas que não faziam parte do diferencial do software, como o cadastro e gerenciamento de usuários.

  Mas a semente e as lições ficaram e, posteriormente, com o advento do projeto Kalibro \cite{of13} como um serviço para extração e análise de métricas estáticas de código fonte, o projeto Mezuro ressurgiu agora como a interface para este serviço.

  Para lidar com os impedimentos encontrados durante o desenvolvimento, foi adotada a plataforma de software livre Noosfero\footnote{\url{http://noosfero.org}}, também feita sobre Ruby on Rails, que traz a confiabilidade e credibilidade de um software nos repositórios oficiais do Linux Debian junto com uma arquitetura de plugins interessante para se estender suas funcionalidades.

  Até julho de 2013 o desenvolvimento se deu com base nesta plataforma. Porém, ela era desestimulante para o time essencialmente em decorrência de as tecnologias que ela força o uso são bastante antigas. Isto implicou em uma série de situações desagradáveis para o time que limitavam muito sua capacidade de inovar e entregar um software de qualidade:

  \begin{itemize}
    \item arcabouço de testes é ultrapassado e com suporte limitado a plugins;
    \item incompatível com as bibliotecas Ruby mais recentes;
    \item incompatível com plugins JavaScript mais atuais e,
    \item baixa performance em produção.
  \end{itemize}

  Além disso, muitas funcionalidades do Noosfero como comunidades, calendário, amizade entre usuários e outras não eram interessantes para o Mezuro e apenas adicionavam mais complexidade ao usuário do sistema.

  Com todos estes problemas e sem ter alcançado uma versão de produção em quase dois anos de desenvolvimento, o time então resolveu abandonar esta versão. O código fonte para esta versão pode ser encontrado em \url{https://github.com/mezuro/noosfero-plugin}.

  \subsection{Estado atual} \label{subsec:estado-atual}
  Neste contexto o time se sentiu confiante, tanto no sentido de conhecimento técnico como conhecimento dos requisitos do software, para então dar uma nova chance a desenvolver um software independente ainda utilizando o arcabouço Ruby on Rails agora em sua quarta versão.

  Para diminuir ainda mais a complexidade no Mezuro, toda a comunicação com o serviço Kalibro foi extraída para uma biblioteca de Ruby denominada kalibro\_gem\footnote{\url{http://rubygems.org/gems/kalibro_gem}}. Esta biblioteca reaproveitou boa boa parte do código gerado na versão anterior. Boa parte do trabalho para lançá-la foi consistiu de criar testes automatizados e corrigir pequenos problemas.

  Da mesma forma, para possibilitar que a equipe focasse nas funcionalidades específicas do software, o máximo de bibliotecas caixa-preta disponíveis em Ruby foi empregada.

  Tudo isso, após seis meses de trabalho, culminou na primeira versão realmente utilizável do software. Seu código fonte pode ser encontrado em \url{https://github.com/mezuro/mezuro}.

  \subsection{Lições aprendidas} \label{subsec:licoes-aprendidas}
  Agora, com base nos conceitos de \ref{subsec:criteriosclassificacao} e \ref{subsec:qualopcaoescolher}, vamos analisar as decisões tomadas para as três versões do projeto e refletir sobre elas.

    \subsubsection{Primeira versão}
    Em face destes conceitos, claramente a decisão de se fazer um software novo logo na primeira versão do Mezuro não foi acertada. Faltava experiência à equipe, tanto para reconhecer que não seria capaz de realizar um projeto tão grande a partir do nada, como conhecimento do arcabouço que estavam adotando. Um maior conhecimento do arcabouço resolveria por exemplo o problema citado do gerenciamento de usuários ao se utilizar a biblioteca de Ruby chamada Devise\footnote{http://rubygems.org/gems/devise} que faz exatamente isso sem esforço algum para o programador.

    \subsubsection{Segunda versão} \label{subsubsec:segundaversao}
    À segunda versão sim cabe uma análise mais detalhada. Quanto a lista de vantagens e desvantagens da tabela \ref{tab:projeto-existente} é interessante notar que o sistema Noosfero se converteu em vantagens nas quatro dimensões. Sendo esse um dos motivos mais fortes para sua adoção.

    Do ponto de vista de abstração, a arquitetura de plugins fornecida pelo Noosfero é excelente, permitindo que praticamente seja desenvolvida uma segunda aplicação com toda a estrutura de uma aplicação Rails.

    Porém, quando analisamos o quesito de seleção, começamos a encontrar a raiz de um dos problemas que nos levou a abandonar esta versão. É possível estender funcionalidades, mas não é possível suprimir funcionalidades que não eram interessantes para nossa realidade e éramos forçados a conviver com elas.

    Da mesma forma, a especialização a princípio dava muito pouca flexibilidade de parâmetros. O que melhorou consideravelmente quando passou a ser possível que plugins tivessem rotas próprias\footnote{No arcabouço Ruby on Rails, rotas são a forma como URLs do navegador são traduzidas em ações programadas na aplicação (\url{http://guides.rubyonrails.org/routing.html})}, mas não ao ponto de resolver o problema. Por exemplo, para validar se um usuário estava conectado à sua conta antes de permitir que este acesse determinadas páginas era preciso concentrar todo o código em um único \textit{controller}\footnote{Parte da arquitetura MVC}.

    Em seguida a integração também nos apresentou problemas. Repetidamente nos encontrávamos em situações onde para ter a melhor solução seria preciso que modificássemos partes do código do próprio Noosfero, o que evitávamos ao máximo. Mas em situações como as referidas rotas foi inevitável.

    Adicionalmente, podemos introduzir mais uma dimensão de análise que chamaremos de adequação tecnológica. Quando se desenvolve um software, quem levanta os requisitos geralmente o faz baseado em suas ideias e em soluções que já viu aplicadas em outros softwares. Porém as tecnologias utilizadas pelo software base podem limitar a capacidade de atender a estes requisitos.

    Uma situação vivida durante o desenvolvimento desta versão foi ao tentarmos utilizar uma biblioteca JavaScript que dependia de uma versão recente da biblioteca JQuery (também JavaScript). Porém nosso software base já incluía por padrão uma versão antiga que causa conflitos se tentássemos incluir uma nova.

    Finalmente, podemos concluir que no momento da escolha foram levados em conta a lista de vantagens que  sistema base Noosfero traria de acordo com a tabela \ref{tab:projeto-existente} e a abstração disponível, mas faltou profundidade para levar em conta os critérios de seleção, especialização e integração.

    \subsubsection{Terceira versão}
    Ainda é cedo para analisar esta versão. Contudo podemos afirmar que em pouco mais de seis meses de trabalho alcançamos as mesmas funcionalidades da versão anterior com uma equipe menor.

    Nesse tempo, um ponto negativo que podemos destacar sobre o uso extensivo de bibliotecas caixa-preta é o custo de mantê-las sempre atualizadas, dado que esse é um processo parcialmente manual e que a instalação das novas bibliotecas toma entre 5 e 10 minutos em cada ambiente. Em vista dessa quantidade de tempo podemos afirmar que este tempo é um custo irrisório dado o benefício de não ser responsabilidade da equipe manter a grande quantidade de código provido pelas bibliotecas.

    Então, até agora podemos apenas dizer que o futuro é promissor e que isso não seria possível sem a experiência adquirida com a versão anterior.

\section{Conclusão} \label{sec:conclusao}
Reúso de software por meio de bibliotecas caixa-preta é uma solução bastante robusta para otimizar os tempos gastos com desenvolvimento e manutenção em projetos de software \cite{hdghi11}, como é evidenciado pela história apresentada do projeto Mezuro (\ref{sec:projeto-mezuro}).

A adoção de um sistema inteiro como base é algo que precisa de uma análise criteriosa das quatro dimensões de abstração, seleção, especialização e integração \cite{k92}. Mais uma quinta dimensão que sugerimos: adequação tecnológica (\ref{subsubsec:segundaversao}). O que é difícil dada a demanda de grande conhecimento do sistema base e conhecimentos técnicos para se tomar uma decisão.

\newpage
\bibliographystyle{sbc}
\bibliography{impacto-software-livre}

\end{document}
