\documentclass[a4paper,10pt]{letter}
\usepackage[T1]{fontenc}
\usepackage[utf8]{inputenc}
\usepackage{lmodern}
\usepackage[brazilian]{babel}
\usepackage{hyperref}

\signature{ Prof. Dr. Fabio Kon }
\address{ Departamento de Ciência da Computação \\ Rua do Matão, 1010 \\ 05508-090 São Paulo  - SP \\ Brasil}

\begin{document}

\begin{letter}{ Diretor de informática \\ Vice-Reitoria Executiva de Administração (VREA) da Universidade de São Paulo}

\opening{ Caro diretor, }

Tomei conhecimento recentemente de que a estrutura de nuvem de computadores da Universidade de São Paulo está operacional e já com alguns usuários obtendo resultados positivos nesta. Assim, gostaria de indagá-lo quanto a possibilidade de aproveitar sua estrutura em prol de um projeto acadêmico que vem sendo desenvolvido sob minha orientação, no Centro de Competência em Software Livre, por alunos do Departamento de Ciência da Computação do Instituto de Matemática e Estatística.

Sobre o projeto em questão, denominado Mezuro, ele envolve desde alunos de doutorado até alunos de inicição científica. Atualmente, tem o apoio do Núcleo de Apoio às Pesquisas em Software Livre (NAPSOL). O projeto está em vias de ser liberado para uso público assim que solucionarmos questões técnicas como a infraestrutura.

Ele pode ser encontrado atualmente no endereço \url{mezuro.org}, sob uma licença Livre. Nesta versão disponível do Mezuro é possível experimentar suas funcionalidades que tem como objetivo difundir métricas de código-fonte e analisar seus padrões de uso. Ainda não o divulgamos publicamente tal endereço, pois o atual servidor não tem capacidade de atender à demanda.

Assim, caso seja possível, os recursos que esperamos poder utilizar são por volta de uma máquina com alto poder de processamento (ao menos 8 processadores e 8GB de meória RAM) para realizar o cálculo de métricas e outras duas de poder médio (com ao menos dois processadores e 4GB de memória RAM), uma para servir o sítio e outra como servidora de banco de dados. Da mesma forma, gostaríamos de poder associar um endereço IP público apenas ao servidor do sítio.

\closing{Atenciosamente,}

\end{letter}

\end{document}
