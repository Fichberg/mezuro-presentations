\documentclass{beamer}
\usepackage[T1]{fontenc}
\usepackage[utf8]{inputenc}
\usepackage{lmodern}
\usepackage[brazil]{babel}

\usetheme{JuanLesPins}

\title{
       \textbf{Como podemos e devemos monitorar nosso código-fonte} \\
       CCSL - IME - USP\\
       FGA - UnB
      }
\author{
        Rafael Manzo
        Paulo Meirelles \\
        Arthur Del Esposte \\
       }

\begin{document}

\maketitle

\section{Introdução}
  \subsection{Métricas}
  \begin{frame}
    \frametitle{O que são métricas?}
    \framesubtitle{}

    \begin{itemize}
      \item Medidas extraídas do software que fornecem informações sobre sua
        \begin{itemize}
          \item complexidade
          \item capacidade de ser compreendido
          \item testabilidade
          \item manutenabilidade
          \item evolução
        \end{itemize}
      \item Dois tipos
        \begin{itemize}
          \item \textbf{Estáticas}: apenas o fazem análise léxica e sintática código-fonte
          \item Dinâmicas\footnote{Por exemplo, cobertura de testes}: necessitam que o código esteja compilado ou seja executado de alguma forma
        \end{itemize}
    \end{itemize}
  \end{frame}

  \begin{frame}
    \frametitle{Exemplos de métricas}
    \framesubtitle{Primitivas}

    Mais comuns:
    \begin{itemize}
      \item Quantidade de classes
      \item Quantidade de métodos
      \item Linhas de código
      \item Acoplamento
    \end{itemize}

    E assim por diante...
  \end{frame}

  \begin{frame}
    \frametitle{Exemplo de métrica}
    \framesubtitle{Compostas}

    Combinação de uma ou mais métricas primitivas

    \begin{itemize}
      \item \textbf{qm} = quantidade de métodos de uma classe
      \item \textbf{loc} = linhas de código da classe
    \end{itemize}

    \textbf{locm} = média de linhas por método = $qm / loc$
  \end{frame}

  \begin{frame}
    \frametitle{Extratores}
    \framesubtitle{}

    \begin{itemize}
      \item \textbf{Ruby}: metric\_fu
      \item \textbf{Python}: CVSAnaly, pylint
      \item \textbf{Java}: Checkstyle, Analizo
      \item \textbf{C/C++}: Analizo
    \end{itemize}
  \end{frame}

\section{Motivação}
\begin{frame}
  \LARGE{\textbf{Motivação}}
\end{frame}

\begin{frame}
  \frametitle{Crise do software}
  \framesubtitle{O problema}

  \begin{itemize}
    \item Termo que surgiu em 1969
    \item A evolução do hardware permite resolver problemas mais difíceis
    \item Mas a interface para este hardware é complexa
    \item As metodologias de desenvolvimento de software não estão preparadas
  \end{itemize}
\end{frame}

\begin{frame}
  \frametitle{Crise do software}
  \framesubtitle{Usando métricas}

  Podemos incorporar métricas de qualidade as metodologias de desenvolvimento como forma de monitorar a qualidade da produção

  Mas, para isso ser viável temos alguns requisitos:
  \begin{itemize}
    \item interface que agrupe as diversas ferramentas disponíveis hoje no mercado
    \item permita seleção e composição de métricas de forma flexível
    \item manutenção de um histórico de evolução
    \item exiba os resultados de forma amigável
  \end{itemize}
\end{frame}

\begin{frame}
  \frametitle{Crise do software}
  \framesubtitle{Quais métricas? Como interpretar?}

  Indo além, o que é bom ou ruim para um valor de métrica?

  Dada uma classe com 100 linhas de código:
  \begin{itemize}
    \item \textbf{C++}: Bom
    \item \textbf{Java}: Razoável
    \item \textbf{Ruby}: Ruim
  \end{itemize}

  \textbf{Não existe um consenso}
\end{frame}

\section{Soluções similares}
\begin{frame}
  \LARGE{\textbf{Soluções similares}}
\end{frame}

\begin{frame}
  \frametitle{Sonar}
  \framesubtitle{}
\end{frame}

\begin{frame}
  \frametitle{Code Climate}
  \framesubtitle{}
\end{frame}

\section{Mezuro}
\begin{frame}
  \frametitle{}
  \framesubtitle{}

  Mezuro
\end{frame}

\begin{frame}
  \frametitle{Ideais}
  \framesubtitle{}
\end{frame}

\begin{frame}
  \frametitle{Breve história}
  \framesubtitle{}
\end{frame}

\begin{frame}
  \frametitle{Arquitetura}
  \framesubtitle{}
\end{frame}

\begin{frame}
  \frametitle{Principais funcionalidades}
  \framesubtitle{}
\end{frame}

  \subsection{Demonstração}
  \begin{frame}
    \frametitle{Criação e processamento de repositório}
    \framesubtitle{Projeto}
  \end{frame}

  \begin{frame}
    \frametitle{Criação e processamento de repositório}
    \framesubtitle{Repositório}
  \end{frame}

  \begin{frame}
    \frametitle{Criação e processamento de repositório}
    \framesubtitle{Exibição de resultados}
  \end{frame}

  \begin{frame}
    \frametitle{Criação de configuração}
    \framesubtitle{Configuração}
  \end{frame}

  \begin{frame}
    \frametitle{Criação e processamento de repositório}
    \framesubtitle{Escolha de métrica e interpretação}
  \end{frame}

\section{Conclusão}
\begin{frame}
  \frametitle{Conclusão}
  \framesubtitle{}

  Conclusão
\end{frame}

\begin{frame}
  \frametitle{Conclusão}
  \framesubtitle{Revisão}
\end{frame}

\begin{frame}
  \frametitle{Conclusão}
  \framesubtitle{Próximos passos}
\end{frame}

\begin{frame}
  \frametitle{Conclusão}
  \framesubtitle{Nos acompanhe}

  Obrigado!
\end{frame}
\end{document}